\documentclass{scrartcl}
\usepackage{chez}

\begin{document}
\section{April 29, 2019}
From the Laplacian in polar coordinates, we get the time independent Schr\"odinger equation
\[
	-\frac{\hbar^2}{2m} \brackets*{\frac{1}{r^2} \pdv{}{r} \parens*{r^2 \pdv{\psi}{r}} + \frac{1}{r^2 \sin \theta} \pdv{}{\theta} \parens*{\sin \theta \pdv{\psi}{\theta}} + \frac{1}{r^2 \sin^2 \theta} \pddv{\psi}{\varphi}} + V \psi = E\psi.
\]
We assume that we have a solution of the form \(\psi(r, \theta, \varphi) = R(r) \cdot Y(\theta, \varphi)\), where \(R\) is called the radial part and \(Y\) is called the angular part. Substituting this in and simplifying, we get
\begin{align*}
	\frac{1}{R} \dv{}{r} \parens*{r^2 \dv{R}{r}} - \frac{2mr^2}{\hbar^2} [V(r) - E] &= \ell(\ell + 1) \\
	\frac{1}{Y} \brackets*{\frac{1}{\sin \theta} \pdv{}{\theta} \parens*{\sin\theta \pdv{Y}{\theta}} + \frac{1}{\sin^2\theta} \pddv{Y}{\varphi}} &= -\ell(\ell + 1)
\end{align*}
for the radial and angular part of the wavefunction respectively, and where \(\ell(\ell + 1)\) is just some constant (motivation provided later). Note that the second equation has no dependence on \(V(r)\), so when we solve this, we can use the solutions for all \(V(r)\).

For the radial equation, if we make the substitution \(u(r) = r R(r)\), then we get the equation
\[
	-\frac{\hbar^2}{2m} \ddv{u}{r} + \brackets*{V(r) + \frac{\hbar^2}{2m}\frac{\ell(\ell + 1)}{r^2}} u = Eu.
\]
If we are looking for spherically symmetric solutions, then the second equation gives us \(\ell = 0\). Simplifying this is nontrivial, so we will stop here.

\begin{example}[Infinite spherical well]
	Suppose we want the spherically symmetric solutions \(\psi(r)\) of the infinite spherical well
	\[
		V(r) = \begin{cases}
			0      & r \leq a \\
			\infty & r > a
		\end{cases}.
	\]
	On the outside of the well, we have \(u(r) = 0\), and on the inside, we have
	\[
		-\frac{\hbar^2}{2m} \ddv{u}{r} = Eu.
	\]
	This has the general solution
	\[
		u(r) = A \sin kr + B \cos kr,
	\]
	where \(k = \frac{\sqrt{2mE}}{E}\). In order for \(\psi = u/r\) to be finite at \(r = 0\), we must have \(B = 0\). Therefore, we have
	\[
		u(r) = A \sin kr, \qquad \psi(r) = \frac{u(r)}{r}.
	\]
	Considering the boundary condition at \(r = a\), we have
	\[
		\sin ka = 0 \implies k_n a = n \pi \implies k_n = \frac{n\pi}{a},
	\]
	and we can compute the normalization constant \(A = \frac{1}{\sqrt{2\pi a}}\).
\end{example}

\begin{example}[Hydrogen Atom]
	Let's find the spherically symmetric solutions to the hydrogen atom potential
	\[
		V(r) = -\frac{Z\mathrm e^2}{r},
	\]
	where \(Z\) is the atomic number and \(\mathrm e\) is the elementary charge. Here we use \vocab{Gaussian units}, i.e.\ \(4\pi \eps_0 = 1\). Let \(r = \sqrt{-\frac{\hbar^2}{8mE}} \rho\) and \(\lambda = \frac{Z\mathrm e^2}{\hbar} \sqrt{-\frac{m}{2E}}\). This gives the equation
	\[
		\ddv{u}{\rho} + \lambda \frac{u}{\rho} - \frac{u}{4} = 0.
	\]
	First, note that as \(\rho \to \infty\), we get \(u(\rho) \sim e^{-\rho/2}\). Assume that
	\[
		u(\rho) = h(\rho) e^{-\rho/2}.
	\]
	Substituting this into the ODE, we get
	\[
		\ddv{h}{\rho} - \dv{h}{\rho} + \frac{\lambda}{\rho} h = 0.
	\]
	Let \(h(\rho) = \sum_{j = 1}^{\infty} b_i \rho^j\), and then we get the recurrence relation
	\[
		b_{j + 1} = -\frac{\lambda - j}{j(j + 1)} b_j.
	\]
	This means that as \(\rho \to \infty\), we have \(b_{j + 1} \sim \frac{b_j}{j + 1}\), which implies that \(h(\rho)\) grows exponentially. However, there is a solution if \(\lambda - j = 0\) at some point, i.e.\ if \(\lambda \in \ZZ_+\).
	
	Let \(\lambda = n\) be an integer. This is called the \vocab{principal quantum number}. Then, we have that
	\[
		n = \lambda = \frac{Z\mathrm e^2}{\hbar} \sqrt{-\frac{m}{2E}} \implies E_n = -\frac{Z^2m\mathrm e^4}{2\hbar} \frac{1}{n^2} = -\frac{1}{2} \frac{Z^2\mathrm e^2}{a_0} \frac{1}{n^2} = \frac{E_1}{n^2},
	\]
	where \(a_0 = \frac{\hbar^2}{m\mathrm e^2}\).
	
	The eigenfunctions are hard to explicitly write, but the ground state is
	\[
		\psi_1(r) = \frac{1}{\sqrt \pi} \parens*{\frac{a_0}{Z}}^{-3/2} e^{-Zr/a_0}.
	\]
	This gives us all spherically symmetric solutions, which correspond to 1s, 2s, 3s, \ldots orbitals.
\end{example}



\end{document}

