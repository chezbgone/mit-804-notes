\documentclass{standalone}
\usepackage{chez}

\begin{document}
\section{February 06, 2019}
\subsection{Logistics}
The course is heavily based off of 8.03\footnote{oops}.
Going to recitation is recommended but not mandatory.
PSets are due Thursdays at 5PM, once per week, and at some point
in the middle of the course it will change to Fridays at 5PM.\@
This class is known for lengthy PSets because of the computational nature
of the class, so start them early.
There will be two midterms and a final.
Grades are weighted as follows:
\begin{itemize}
	\item Problem Sets: 20\%
	\item Midterm 1: 20\%
	\item Midterm 2: 25\%
	\item Final: 35\%
\end{itemize}
There are no textbooks required for this course,
but there are a few recommended texts.

Lectures will be on the blackboard, no slides.
The advantage to this is that the lecture will be
so slow that students can follow along.
As per MIT-time, we will typically start lectures at 14:35 and end at 15:55.

Also, if you're reading these notes, and want to skip
the experimental motivation, you can skip to
this \hyperref[subsect:rigorous]{part} where we start doing things rigorously.

\subsection{Introduction/Motivation}
Quantum mechanics was discovered because people started to realize that
classical physics was not a sufficient description of our universe.
We will discuss some of the experiments that display this
to motivate quantum mechanics.

\iffalse
\begin{itemize}
	\item blackbody radiation
	\item double slit
	\item photoelectric effect
	\item Compton effect
\end{itemize}
\fi

\subsubsection{Photoelectric Effect}
Consider the experiment where we shine light on a metallic surface. Then, it turns out that this surface emits elections, called \vocab{photoelectrons}. This was discovered by Hertz in 1887. The energies of the photoelectrons range from \(0\) to some maximum.

In 1916, Millikan performed an experiment with the following setup:
\begin{center}
	\begin{asy}
		import cse5;
		size(5cm);
		
		guide squiggly(path g, real stepsize, real slope=45){
			real len = arclength(g);
			real step = len / round(len / stepsize);
			guide squig;
			for (real u = 0; u < len; u += step){
				real a = arctime(g, u);
				real b = arctime(g, u + step / 2);
				pair p = point(g, a);
				pair q = point(g, b);
				pair np = unit( rotate(slope) * dir(g,a));
				pair nq = unit( rotate(0 - slope) * dir(g,b));
				squig = squig .. p{np} .. q{nq};
			}
			squig = squig .. point(g, length(g)){unit(rotate(slope)*dir(g,length(g)))};
			return squig;
		}

		D(box((-3.7, 9), (-2.7, 5)));
		D(box((3.7, 9), (2.7, 5)));
		D((-3.7, 7)--(-6, 7)--(-6, 0)--(-0.5, 0));
		D((3.7, 7)--(6, 7)--(6, 0)--(0.5, 0));
		D((-0.5, 0.6)--(-0.5, -0.6));
		D((0.5, 1)--(0.5, -1));

		filldraw(CR((6, 3.5), 0.5), white, black);
		D(shift(6, 3.5)*(dir(-135)--dir(45)), EndArrow(TeXHead));
		MP("I", (6.7, 3.5), dir(0));

		D(squiggly((-0.5, 9.2)--(-2.3, 7.1), 0.5, 30)--(-2.4, 7), EndArrow(TeXHead));
		label("light", (-0.5, 9.2), dir(60));
		D((-2.4, 6.5)--(-0.4, 6.5), EndArrow(TeXHead));
		label("\(\mathrm{e^-}\)", (-0.4, 6.5), dir(-30));

		MP("V", (0, 1));
		MP("C", (-3.2, 5), dir(-90));
		MP("A", (3.2, 5), dir(-90));
	\end{asy}
\end{center}

It was discovered that if \(V > 0\), then the electrons \(\mathrm{e^-}\) were attracted to A, and if \(V < 0\), they were repelled. If the stopping potential is \(V = V_0\), and we plot the stopping potential with respect to the current measured, we get something like this:
\begin{center}
	\begin{asy}
		import cse5;
		import graph;

		D((-3, 0)--(5, 0), gray(0.6), Arrows(TeXHead));
		D((0, 0)--(0, 4), gray(0.6), EndArrow(TeXHead));

		D(graph(new real(real t){ return 1.2 * sqrt(t + 2); }, -2, 5),
			EndArrow(TeXHead)
		);

		MP("I", (0, 4), dir(0));
		MP("V", (5, 0), dir(90));
		MP("V_0", (-2, 0), dir(-90));
	\end{asy}
\end{center}

We know that the energy of the electron is
\begin{align*}
	K_{\text{max}} &= e V_0 \\
		&= \half m_e V_{\text{max}}^2
\end{align*}
It was found that if we changed the intensity of the light, the stopping potential does not change with it, which is not what classical mechanics predicts. Other observations include:
\begin{itemize}
	\item \(K_{\text{max}}\) is independent of intensity of the light.
	\item \(K_{\text{max}}\) increases with frequency \(\nu\).
	\item There is a minimum frequency \(\nu_0\) below which no \(\mathrm{e^-}\) are detected.
\end{itemize}

In 1905, Einstein realized the following:
\begin{itemize}
	\item Light comes in photons with energy \(E = h \nu\), where \(h\) is Planck's constant.
	\item Electrons are bound to metal with some binding energy \(W \geq W_0\).
	\item Each photoelectron is ejected by one photon.
	\item The energy of the ejected electron is \(K = h\nu - W\), and therefore \(K_{\text{max}} = h\nu - W_0\).
\end{itemize}

Overall, this experiment shows that:
\begin{itemize}
	\item Increasing the intensity of the light only increases the number of photoelectrons.
	\item The minimum frequency \(\nu_0\) satisfies
	\[
		K_{\text{max}} = h\nu - W_0 = h (\nu - \nu_0) \implies \nu_0 = \frac{W_0}{h}.
	\]
\end{itemize}

\subsubsection{Compton Effect}
Consider the experiment where we shine an x-ray into a graphite object. It is known that an electron is emitted, and the x-rays are scattered at many angles. We will place a detector that detects photons at a fixed angle \(\theta\) for now.
\begin{center}
	\begin{asy}
		import cse5;
		size(10cm);

		guide squiggly(path g, real stepsize, real slope=45){
			real len = arclength(g);
			real step = len / round(len / stepsize);
			guide squig;
			for (real u = 0; u < len; u += step){
					real a = arctime(g, u);
					real b = arctime(g, u + step / 2);
					pair p = point(g, a);
					pair q = point(g, b);
					pair np = unit( rotate(slope) * dir(g,a));
					pair nq = unit( rotate(0 - slope) * dir(g,b));
					squig = squig .. p{np} .. q{nq};
			}
			squig = squig .. point(g, length(g)){unit(rotate(slope)*dir(g,length(g)))};
			return squig;
		}

		D(squiggly((-1.5, 0)--(0, 0), 0.1, 30));
		D(squiggly((0, 0)--(0.7 * dir(60)), 0.1, 30));
		D((0, 0)--dir(-30), EndArrow(TeXHead));
		D((0, 0)--(0.5, 0), gray(0.6));

		D(shift(0.7 * dir(60)) * rotate(60) * ((0, -0.1)--(0, 0.1)));
		D(CP(0.62 * dir(60), shift(0.7 * dir(60)) * rotate(60) * (0, 0.1), 0, 120));

		filldraw(box((-0.1, -0.1), (0.1, 0.1)), white, black);

		MA("\theta", dir(0), (0, 0), dir(60), 0.2);
		//MA("\varphi", dir(-30), (0, 0), dir(0), 0.3);

		label(minipage("\centering x-ray\\\(\nu_0, \lambda_0\)"), (-1.5, 0), dir(90));
		label("detector", 0.7 * dir(60) + (0.1, 0), dir(0));
		MP("e^-", dir(-30), dir(0));
	\end{asy}
\end{center}

Classically, we expect all of the detected photons to have wavelength the same as the original photons. However, if (for a specific \(\theta\)) we measure the distribution of photons with each wavelength, we get the following:
\begin{center}
	\begin{asy}
		import cse5;
		size(5cm);

		D((0, 0)--(0, 5), gray(0.4), EndArrow(TeXHead));
		D((0, 0)--(7, 0), gray(0.4), EndArrow(TeXHead));
		
		real e = 2.718281828;
		D(graph(new real(real t){
			return 3 e^(-3(t - 2)^2);
		}, 0.5, 3.4));
		D((2, 0.1)--MP("\lambda_0", (2, -0.1), dir(-90)));

		MP("I", (0, 5), dir(45));
		MP("\lambda", (7, 0), dir(45));
		MP("\theta = 0^\circ", (6, 5), dir(-135));
	\end{asy}
	\quad
	\begin{asy}
		import cse5;
		size(5cm);

		D((0, 0)--(0, 5), gray(0.4), EndArrow(TeXHead));
		D((0, 0)--(7, 0), gray(0.4), EndArrow(TeXHead));
		
		D((1, 0)..(1.25, 0.1)..(1.5, 0.4)..(1.75, 1.5)..controls (1.825, 2.8)..(2, 3)..(2.3, 2.9)..controls (2.5, 2.9) and (2.7, 3.5)..(2.8, 3.7)--(2.8, 3.7)..(3, 1.8)..(3.8, 0.2)..(4, 0.1)..(4.5, 0));
		D((2, 0.1)--MP("\lambda_0", (2, -0.1), dir(-90)));

		MP("I", (0, 5), dir(45));
		MP("\lambda", (7, 0), dir(45));
		MP("\theta = 90^\circ", (6, 5), dir(-135));
	\end{asy}

	\begin{asy}
		import cse5;
		import graph;
		size(5cm);

		D((0, 0)--(0, 5), gray(0.4), EndArrow(TeXHead));
		D((0, 0)--(7, 0), gray(0.4), EndArrow(TeXHead));
		
		D(graph(new real(real t){
			real e = 2.718281828;
			return 1.7 e^(-7(t - 2)^2) + 3 e^(-3(t - 3.2)^2);
		}, 1, 5));
		D((2, 0.1)--MP("\lambda_0", (2, -0.1), dir(-90)));

		MP("I", (0, 5), dir(45));
		MP("\lambda", (7, 0), dir(45));
		MP("\theta = 135^\circ", (7, 5), dir(-135));
	\end{asy}
	\quad
	\begin{asy}
		import cse5;
		import graph;
		size(5cm);

		D((0, 0)--(0, 5), gray(0.4), EndArrow(TeXHead));
		D((0, 0)--(7, 0), gray(0.4), EndArrow(TeXHead));
		
		real e = 2.718281828;
		D(graph(new real(real t){
			return 2 e^(-5(t - 2)^2) + 3.5 e^(-2(t - 4.2)^2);
		}, 1, 6));
		D((2, 0.1)--MP("\lambda_0", (2, -0.1), dir(-90)));

		MP("I", (0, 5), dir(45));
		MP("\lambda", (7, 0), dir(45));
		MP("\theta = 45^\circ", (7, 5), dir(-135));
	\end{asy}
\end{center}

It is not clear how we get the second peak! Let's look at a more idealized version of the experiment. Let the incoming photon have energy \(E_0\) and momentum \(p_0\), the emitted electron have total energy \(E\), kinetic energy \(K\), and momentum \(p\), and the scattered photon have energy \(E_1\) and momentum \(p_1\).
\begin{center}
	\begin{asy}
		import cse5;
		size(10cm);

		guide squiggly(path g, real stepsize, real slope=45){
			real len = arclength(g);
			real step = len / round(len / stepsize);
			guide squig;
			for (real u = 0; u < len; u += step){
				real a = arctime(g, u);
				real b = arctime(g, u + step / 2);
				pair p = point(g, a);
				pair q = point(g, b);
				pair np = unit( rotate(slope) * dir(g,a));
				pair nq = unit( rotate(0 - slope) * dir(g,b));
				squig = squig .. p{np} .. q{nq};
			}
			squig = squig .. point(g, length(g)){unit(rotate(slope)*dir(g,length(g)))};
			return squig;
		}

		D(squiggly((-1.5, 0)--(0, 0), 0.1, 30));
		D(squiggly((0, 0)--dir(60), 0.1, 30)--(1.01 * dir(60)), EndArrow(TeXHead));
		D((0, 0)--dir(-30), EndArrow(TeXHead));
		D((0, 0)--(0.5, 0), gray(0.6));

		filldraw(box((-0.1, -0.1), (0.1, 0.1)), white, black);

		MA("\theta", dir(0), (0, 0), dir(60), 0.2);
		MA("\varphi", dir(-30), (0, 0), dir(0), 0.3);

		label("photon \(E_0, p_0\)", (-1.5, 0), dir(90));
		label(minipage("\centering scattered photon\\\(E_0, p_0\)"), shift(-0.2, 0)*dir(60), dir(0));
		MP("e^- \; K, E, p", dir(-30), dir(0));
	\end{asy}
\end{center}

By conservation of momentum, we have
\begin{align*}
	p_0 &= p_1 \cos\theta + p \cos\varphi \\
	p_1 \sin\theta &= p \sin\varphi.
\end{align*}
Subtracting \(p_1 \cos\theta\) in the first equation, squaring the equations and adding, we have
\[
	p_0^2 + p_1^2 - 2 p_0 p_1 \cos\theta = p^2. \label{eqn:compton-momentum}\tag{A}
\]

Now by conservation of energy, we have
\[
	E_0 + m_e c^2 = E_1 + \underbracket{K + m_e c^2}_E \implies E - m_e c^2 = E_0 - E_1. \label{eqn:compton-energy}\tag{B}
\]

From special relativity, we have that
\[
	E^2 = (pc)^2 + (mc^2)^2,
\]
and that for a photon, \(p = E/c = h/\lambda\).

Multiplying \cref{eqn:compton-momentum} by \(c^2\), we have
\[
	E_0^2 + E_1^2 - 2E_0 E_1 \cos\theta = (pc)^2 = E^2 - (m_e c^2)^2.
\]
Squaring \cref{eqn:compton-energy}, we have
\[
	E^2 + (m_e c^2)^2 - 2E(m_e c^2) = E_0^2 + E_1^2 - 2E_0 E_1.
\]
Subtracting these two equations, we have
\[
	2E_0E_1(1 - \cos\theta) = 2E(m_e c^2) - 2(m_e c^2)^2.
\]
Since \(E = K + m_e c^2 = (E_0 - E_1) + m_e c^2\), this simplifies to
\[
	E_0E_1(1 - \cos\theta) = (E_0 - E_1)(m_e c^2).
\]
Therefore,
\[
	E_0 - E_1 = \frac{hc}{\lambda_0} - \frac{hc}{\lambda_1} = \frac{E_0E_1}{m_e c^2}(1 - \cos\theta) = \frac{h^2c^2}{\lambda_0 \lambda_1 m_e c^2} (1 - \cos\theta)
		\implies \lambda_0 - \lambda_1 = \Delta \lambda = \frac{h}{m_e c^2}(1 - \cos\theta).
\]








\end{document}

