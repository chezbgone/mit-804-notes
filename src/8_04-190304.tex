\documentclass{scrartcl}
\usepackage{chez}

\begin{document}
\section{March 04, 2019}
\subsection{On the Separable Solutions}
Consider the solutions to \(\hat H\) in the time independent Schr\"odinger equation. They are \vocab{stationary states}, i.e.\ the observables are all independent of time for these states. In particular, we have the probability
\[
	P(x) = \abs{\Psi(x, t)^2} = \abs{\psi(x) \cdot c e^{-iEt/\hbar}}^2 = \abs{\psi(x) c}^2.
\]
Similarly, the expected value of some observable \(\hat A\) is
\[
	\angles{A} = \int \Psi^*(x, t) \hat A \Psi(x, t) \,dx = \int \psi^*(x) c^* \hat A \psi(x) c \,dx.
\]

Moreover, these states have a definite energy, i.e.\ \(\Delta E = 0\). We can indeed compute
\begin{align*}
	\angles H &= \int \psi^*(x) \hat H \psi(x) \,dx = E \\
	\angles{H^2} &= \int \psi^*(x) \hat H^2 \psi(x) \,dx = E^2,
\end{align*}
since \(\hat H \psi = E \psi\). Therefore, we have \(\Delta E = \angles{H^2} - \angles H^2 = 0\).

\begin{proposition}[Separable Solutions to the Schr\"odinger Equation]
	The most important thing about the separable solutions is that a general solution is a linear combination of the separable solutions. This is because we found all of the eigenfunctions \(\psi_n(x)\) of a Hermitian operator \(\hat H\), which we know form a basis for the entire space. Therefore, once we find the eigenfunctions \(\psi_n(x)\), a general solution can be written as
	\[
		\Psi(x, t) = \sum_{n = 1}^{\infty} c_n \psi_n(x) e^{-i E_n t / \hbar}.
	\]
\end{proposition}

\subsection{Infinite Square Well}
Consider a particle with the potential
\[
	V(x) = \begin{cases}
		0 & 0 \leq x \leq a \\
		\infty & \text{otherwise}.
	\end{cases}
\]
\begin{center}
	\begin{asy}
		import cse5;
		size(5cm);

		D((-1, 0)--(4, 0), gray(0.5), Arrows(TeXHead));
		D((0, -0.2)--(0, 3), gray(0.5), EndArrow(TeXHead));

		D((0, 2.8)--(0, 0)--(3, 0)--(3, 2.8), linewidth(1));

		MP("x", (4, 0), dir(0));
		MP("a", (3, -0.2), dir(-90));
		MP("0", (0, -0.2), dir(-90));
		MP("V(x)", (0, 3), dir(90));
	\end{asy}
\end{center}
This represents a particle that is completely free except at the two ends at \(x = 0\) and \(x = a\). Let the system start at \(\Psi(x, 0) = \psi_0(x)\). Then if we have the eigenstates \(\phi_n(x)\) defined by \(\hat H \phi_n = E_n \phi_n\), we have the general solution
\[
	\Psi(x, t) = \sum_{n} c_n \phi_n(x) e^{-i E_n t / \hbar},
\]
where we find \(c_n\) by taking the dot product at \(t = 0\):
\begin{align*}
	c_n &= \braket{\phi_n}{\psi_0} \\
		&= \int \phi_n^*(x) \phi_0(x) \,dx.
\end{align*}

We will now find the eigenstates. We have the equation
\[
	\hat H = \frac{\hat p^2}{2m} + V(x) = -\frac{\hbar^2}{2m} \ddv{}{x} + V(x)
		\implies -\frac{\hbar^2}{2m} \ddv{\psi(x)}{x} + V(x) \psi(x) = E \psi(x).
\]
Outside of the interval \([0, a]\), we have \(V = \infty\), so in order for \(E\) to be finite, \(\psi(x) = 0\).

Inside the interval, we have something more interesting. In particular, we have
\[
	-\frac{\hbar^2}{2m} \ddv{\psi(x)}{x} = E \psi(x)
		\implies \psi''(x) = -\frac{2mE}{\hbar^2} \psi(x).
\]
If we let \(k = \sqrt{2mE}/\hbar\), then we have the general solution
\[
	\psi(x) = A \sin kx + B \cos kx.
\]
To find \(A\) and \(B\), we can use the fact that the wavefunctions are normalized, and they are continuous, i.e.\ \(\psi(0) = \psi(a) = 0\). In particular,
\[
	\psi(0) = A \sin 0 + B \cos 0 = 0 \implies B = 0.
\]
Moreover,
\[
	\psi(a) = A \sin ka = 0 \implies \sin ka = 0,
\]
because otherwise \(A = 0\) would give us the zero wavefunction. This means \(ka = n \pi\) for some \(n \in \ZZ\). Then we have the solutions \(k = k_n = \frac{n\pi}{a}\) for \(n \in \ZZ\). Note that \(n = 0\) is bad because we get the zero wavefunction, and \(n < 0\) give us repeated solutions, so we will just take \(n \in \NN\). Surprisingly, this does not give a restriction on \(A\), but rather a restriction on \(E\). In particular, we get the possible values of \(E\) are
\[
	E_n = \frac{\hbar^2 k_n^2}{2m} = \frac{n^2 \pi^2 \hbar^2}{2 m a^2}.
\]
This means that the particle can only have specific energies! We say that the particle is \vocab{quantized}.

Finishing up, we normalize to get \(\abs{A}^2 = \frac{2}{a}\), and for simplicity we will choose the real value \(A = \sqrt{\frac{2}{a}}\). This gives us the eigenstates
\[
	\phi_n(x) = \sqrt{\frac{2}{a}} \sin(k_n x),
\]
where \(k_n = \frac{n\pi}{a}\) for \(n \in \NN\), yielding \(E_n = \frac{\hbar^2 k_n^2}{2m}\). Note that \(E_n > 0\). This makes sense because we have a range that \(x\) is in, meaning that \(\Delta x\) is finite. This means \(\Delta p\) is nonzero, which means \(E\) is nonzero, because otherwise \(p\) will have a certain value.










\end{document}
