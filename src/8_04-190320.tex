\documentclass{scrartcl}
\usepackage{chez}

\begin{document}
\section{March 20, 2019}
Consider the finite square well with potential \(V(x) = \begin{cases} -V_0 & \abs x < a \\ 0 & \text{otherwise} \end{cases}\). Suppose we have an energy \(-V_0 < E < 0\).

Consider the region where \(x < -a\). By the Schr\"odinger equation,
\[
	-\frac{\hbar^2}{2m} \ddv{\psi(x)}{x} = E\psi(x) \implies \psi'' = \alpha^2 \psi, \qquad \text{where \(\alpha = \frac{\sqrt{-2mE}}{\hbar}\)}.
\]
Then, the general solution in this region is
\[
	\psi(x) = A e^{-\alpha x} + B e^{\alpha x}.
\]
Since \(\psi\) must be normalizable, we must have \(A = 0\), and \(\boxed{\psi(x) = B e^{\alpha x}}\).  For the region where \(a < x\), the situation is similar to when \(x < -a\), and we have the general solution
\[
	\boxed{\psi(x) = F e^{-\alpha x}}.
\]

In the region where \(-a < x < a\), we have
\[
	-\frac{\hbar^2}{2m} \ddv{\psi(x)}{x} - V_0 \psi(x) = E\psi(x) \implies \psi'' = -k^2 \psi, \qquad \text{where \(k = \frac{\sqrt{2m(E + V_0)}}{\hbar}\)}.
\]
The general solution for this is
\[
	\boxed{\psi(x) = C \sin kx + D \cos kx}.
\]

Now we have to patch these solutions together. To do this, we will use the fact that \(\psi\) and \(\psi'\) are continuous, and also the fact that \([\hat H, \hat \pi] = 0\). We will look for the even solutions. We have
\[
	\psi_{\text{even}}(x) = \begin{cases}
		F e^{\alpha x} & x < -a \\
		D \cos kx & -a < x < a \\
		F e^{-\alpha x} & x > a.
	\end{cases}
\]
Now we check the boundary. At \(x = a\), we have
\begin{align*}
	\psi(a) &= F e^{-\alpha a} = D \cos ka \\
	\psi'(a) &= -\alpha F e^{-\alpha a} = -kD \sin ka.
\end{align*}
Dividing, we have
\[
	\alpha = k \tan ka.
\]

Letting \(p = ka\) and \(q = \alpha a\), we can rewrite the equation as
\[
	q = p \tan p.
\]
We also know that
\[
	p^2 + q^2 = a^2 \cdot \frac{2mV_0}{\hbar^2} \eqqcolon R_0^2.
\]

If we plot these two equations, we get something like the following:
\begin{center}
	\begin{asy}
		import cse5;
		size(6cm);

		real xmin = 0, xmax = 12;
		real ymin = 0, ymax = 12;
		real pi = 4 * atan(1);
		real eps = 0.0001;

		real ptanp(real p){ return p * tan(p); }
		D(graph(ptanp, 0, pi/2 - eps), n_orange);
		D(graph(ptanp, pi + eps, 3 * pi/2 - eps), n_orange);
		D(graph(ptanp, 2 * pi + eps, 5 * pi/2 - eps), n_orange);
		D(graph(ptanp, 3 * pi + eps, 7 * pi/2 - eps), n_orange);
		D(graph(ptanp, 4 * pi + eps, 9 * pi/2 - eps), n_orange);

		real R0 = 10;
		D(CR((0, 0), R0, 0, 90), n_blue);

		clip(box((xmin, ymin), (xmax, ymax)));
		D((xmin, 0)--MP("p", (xmax, 0), dir(0)), gray(0.6), EndArrow(TeXHead));
		D((0, ymin)--MP("q", (0, ymax), dir(90)), gray(0.6), EndArrow(TeXHead));
	\end{asy}
\end{center}
The blue curve is the equation \(p^2 + q^2 = R_0^2\), and the orange curves are the equation \(q = p \tan p\). We take the intersection points to be the solutions.


\end{document}

