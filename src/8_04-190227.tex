\documentclass{scrartcl}
\usepackage{chez}

\begin{document}
\section{February 27, 2019}
\subsection{Measurement and Wavefunction Collapse}
\begin{proposition}[Spectral Theorem]
	Given a Hermitian operator \(\hat A\), we have the following:
	\begin{itemize}
		\item The eigenvalues are real.
		\item The eigenfunctions are orthogonal, i.e.\ \(\angles{\psi_n, \psi_m} = \delta_{n, m}\) for eigenfunctions \(\psi_n, \psi_m\).
		\item The eigenfunctions form a complete basis, i.e.\ for all \(f(x)\), there exists a function \(c(n)\) such that
		\[
			f(x) = \int c(n) \psi_{A, n}(x) \,dn.
		\]
	\end{itemize}
\end{proposition}

When we measure an observable, it causes the wavefunction to collapse to one of its eigenfunctions. The value of \(\abs{c(n)}^2\) gives the corresponding probability of which eigenvalue we will measure if we collapse the wavefunction.


\begin{example}
	Consider the momentum operator \(\hat p\). Recall that the eigenfunctions of \(\hat p\) are \(f\colon x \mapsto e^{ikx}\) for \(k \in \RR\). Then we have that these functions form a complete basis of the vector space of wavefunctions, i.e.\
	\[
		\psi(x) = \int c(k) e^{ikx} \,dk.
	\]
\end{example}

To compute the \(c(n)\), note that we can take the inner product of the wavefunction with the corresponding eigenfunction. This works because of both orthogonality and the wavefunctions form a complete basis:
\begin{align*}
	\angles{\psi_{A, n}(x), \psi(x)}
		&= \angles*{\psi_{A, n}(x), \int c(m) \psi_{A, m}(x) \, dm} \\
\intertext{Since the inner product is linear,}
		&= \int c(m) \angles*{\psi_{A, n}(x), \psi_{A, m}(x)} \,dm \\
		&= \int c(m) \delta_{n, m} \,dm \\
		&= c(n).
\end{align*}

\subsection{Dirac Notation}
Note that \(\psi(x), \tilde\psi(k), \tilde\psi(p)\) are just different representations of the same wavefunction in different bases. We will consider the basis independent notation, where we write the wavefunction as \(\ket{\psi}\), called a \vocab{ket vector}. Similarly, for \(\psi^*\), we will represent this as \(\bra{\psi}\), called a \vocab{bra vector} in the dual space. If we put these two together, then we get the scalar product
\[
	\braket{\varphi}{\psi} = \angles{\varphi(x), \psi(x)}.
\]
Similarly, we can rewrite the relations
\begin{itemize}
	\item \(\angles A = \angles{\psi(x), \hat A \psi(x)} = \bra{\psi} \hat A \ket{\psi}\)
	\item \(\hat A^\dagger = \hat A \implies \braket{\hat A \varphi}{\psi} = \bra{\varphi}\hat A\ket{\psi}\)
	\item \(\braket{\varphi}{a\psi} = a \braket{\varphi}{\psi}\)
	\item \(\braket{a\varphi}{\psi} = a^* \braket{\varphi}{\psi}\)
\end{itemize}

\subsection{Time Evolution and Schr\"odinger Equation}
There is no rigorous derivation to get Schr\"odinger equation from classical mechanics, otherwise classical physics will already be a quantum theory. However, we can give motivation for it.

\subsubsection{Motivation}
Recall the plane wave
\[
	\psi_p(x, t) = e^{i(kx - \omega t)}.
\]
(We write \(\psi_p\) because it is an eigenfunction of \(\hat p\).) The photoelectric effect tells us that there is a relationship between the frequency and the energy. Therefore, we assume that \(E = h\nu = \frac{h}{2\pi} \omega = \hbar \omega\). We want to get this value with the momentum operator somehow. We get
\[
	i\hbar \dv{}{t} \psi_p(x, t) = \hbar \omega \psi_p(x, t) = E \psi_p(x, t).
\]
But from classical physics, we have \(E = \frac{p^2}{2m} + V\), where \(V\) is the potential energy. This is often called the Hamiltonian of the system. If we define the \vocab{Hamiltonian operator} to be
\[
	\hat H = \frac{\hat p^2}{2m} + \hat V = \frac{\hat p \hat p}{2m} + \hat V,
\]
we have \vocab{Schr\"odinger equation}
\[
	i \hbar \pdv{}{t} \psi(x, t) = \hat H \psi(x, t) = \parens*{\frac{\hat p^2}{2m} + \hat V} \psi(x, t).
\]
If we expand this out in position space, we get
\[
	i\hbar \pdv{}{t} \psi(x, t) = -\frac{\hbar^2}{2m} \pddv{}{x} \psi(x, t) + V(x, t) \psi(x, t).
\]
Here the \(\hat V\) in position space happens to be just multiplication by the potential \(V\).


\subsection{Time-independent Schr\"odinger Equation}
Schr\"odinger equation seems pretty hard to solve, so we will simplify it by assuming that \(V\) is time independent, i.e.\ \(V = V(x)\). Let's try to find a solution of the form
\[
	\Psi(x, t) = \psi(x) \cdot f(t).
\]
This seems like a significant restriction at first, but it turns out that we can construct the general solution with the subset of solutions that we get this way. We compute the derivatives as
\[
	\pddv{\Psi}{x} = \ddv{\psi}{x} f,
		\qquad \pdv{\Psi}{t} = \psi \dv{f}{t}.
\]
Substituting this into Schr\"odinger Equation, we get
\[
	-\frac{\hbar^2}{2m} \ddv{\psi(x)}{x} f(t) + V \psi(x) f(t) = i\hbar \psi(x) \dv{f(t)}{t}.
\]
Dividing by \(\psi(x) f(t)\),
\[
	-\frac{\hbar^2}{2m} \frac{1}{\psi(x)} \ddv{\psi(x)}{x} + V(x) = i\hbar \frac{1}{f(t)} \dv{f(t)}{t}.
\]
Note that the left only depends on \(x\) and the right only depends on \(t\). Therefore, both sides are constant. Let this constant be \(E\).

For the right side, we get
\[
	i\hbar \frac{1}{f(t)} \dv{f(t)}{t} = E \implies f(t) = c e^{-iEt/\hbar} = c e^{-i\omega t}
\]
for some constant \(c\) and where \(\omega = E/\hbar\).

For the left side, we get
\[
	-\frac{\hbar^2}{2m} \frac{1}{\psi(x)} \ddv{\psi(x)}{x} + V(x) = E
\]
multiplying by \(\psi(x)\), we have
\[
	-\frac{\hbar^2}{2m} \ddv{\psi(x)}{x}  + V(x) \psi(x) = E \psi(x),
\]
which is just the equation \(\hat H \psi = E \psi\). Therefore, solving the time-independent Schr\"odinger equation is reduced to finding the eigenfunction of \(\hat H\).








\end{document}

