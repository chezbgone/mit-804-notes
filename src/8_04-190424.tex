\documentclass{scrartcl}
\usepackage{chez}

\begin{document}
\section{April 24, 2019}
\subsection{Three-dimensional Systems}
We will first generalize the Schr\"odinger equation to three dimensions. To do this, we have
\[
	\hat H = \parens*{\frac{\hat p_x^2 + \hat p_y^2 + \hat p_z^2}{2m}} + V(x, y, z, t) = \frac{\hat p^2}{2m} + V(x, y, z, t),
\]
where \(\hat p_x = -i\hbar \pdv{}{x}\), \(\hat p_y = -i\hbar \pdv{}{y}\), \(\hat p_z = -i\hbar \pdv{}{z}\), and we define \(\hat p = \hat p_x^2 + \hat p_y^2 + \hat p_z^2\). Then, the Schr\"odinger equation is
\[
	\hat H \psi = i \hbar \pdv{}{t} \psi.
\]
Sometimes, we will write
\[
	-\frac{\hbar^2}{2m} \nabla^2 \psi(\vec r, t) + V(\vec r, t) \psi(\vec r, t) = i \hbar \pdv{}{t} \psi(\vec r, t).
\]
We will similarly assert that
\begin{gather*}
	P(x, y, z, t) = \abs{\psi(\vec r, t)}^2 \\
	1 = \int \abs{\psi(x, t)}^2 \,d^3r.
\end{gather*}

As in the one-dimensional case, we have the time independent Schr\"odinger equation \(\hat H \psi(\vec r) = E \psi(\vec r)\), and the general solution
\[
	\psi_n(\vec r, t) = \psi_n(\vec r) e^{-i E_n t/\hbar},
\]
where \(\hat H \psi_n = E_n \psi_n\) describes the eigenvectors.

\subsubsection{Free Particle}
Let \(V(\vec r) = 0\). Then we have
\[
	\hat H = \frac{\hat p^2}{2m} = -\frac{\hbar^2}{2m} \nabla^2 = -\frac{\hbar^2}{2m} \parens*{\pddv{}{x} + \pddv{}{y} + \pddv{}{z}},
\]
and we wish to solve
\[
	-\frac{\hbar^2}{2m} \parens*{\pddv{}{x} + \pddv{}{y} + \pddv{}{z}} \psi(\vec r) = E \psi(\vec r).
\]
Let's guess a solution of the form \(\psi(x, y, z) = \psi_x(x) \psi_y(y) \psi_z(z)\). Then we wish to solve
\[
	E \psi_x \psi_y \psi_z = -\frac{\hbar^2}{2m} (
		\psi_y \psi_z \partial_x^2 \psi_x +
		\psi_z \psi_x \partial_y^2 \psi_y +
		\psi_x \psi_y \partial_z^2 \psi_z
	).
\]
Dividing by \(\psi_x \psi_y \psi_z\), we have
\[
	E = -\frac{\hbar^2}{2m} \parens*{
		\frac{\partial_x^2 \psi_x}{\psi_x} +
		\frac{\partial_y^2 \psi_y}{\psi_y} +
		\frac{\partial_z^2 \psi_z}{\psi_z}
	}.
\]
Reducing, we get
\begin{align*}
	-\frac{\hbar^2}{2m}\pddv{}{x} \psi_x = E_x \psi_x &&
	-\frac{\hbar^2}{2m}\pddv{}{y} \psi_y = E_y \psi_y &&
	-\frac{\hbar^2}{2m}\pddv{}{z} \psi_z = E_z \psi_z.
\end{align*}
This is just the equation that describes a one-dimensional free particle! This means that we have
\begin{align*}
	\psi_x(x) = e^{i k_x x} &&
	\psi_y(y) = e^{i k_y y} &&
	\psi_z(z) = e^{i k_z z},
\end{align*}
where
\begin{align*}
	k_x^2 = \frac{2m}{\hbar^2} E_x &&
	k_y^2 = \frac{2m}{\hbar^2} E_y &&
	k_z^2 = \frac{2m}{\hbar^2} E_z.
\end{align*}
Combining, we get
\[
	E = E_x + E_y + E_z = \frac{\hbar^2}{2m} (k_x^2 + k_y^2 + k_z^2) = \frac{\hbar^2}{2m} \vec k^2,
\]
where \(\vec k = (k_x, k_y, k_z)\). Then, we have the full solution
\[
	\psi(\vec r) = e^{i(\vec k \cdot\vec r)} \implies \psi(\vec r, t) = e^{i \vec k \cdot \vec r - \omega t},
\]
where \(\omega = E/\hbar\).

\subsubsection{Infinite Cube Well}
Let \(V(\vec r) = \begin{cases}
	0 & 0 < x, y, z < a \\ \infty & \text{else}
\end{cases}\). We will again use separation of variables. Recall that in one dimension, we have the eigenstates
\[
	\phi_n(x) = \sqrt{\frac{2}{a}} \sin(k_n x), \qquad E_n = \frac{\hbar^2}{2m} k_n^2, \qquad k_n = \frac{(n + 1)\pi}{a},
\]
where \(n = 0, 1, 2, \dots\). Then, in three-dimensions, we have the eigenstates
\[
	\phi_n(\vec r) = \brackets*{\sqrt{\frac{2}{a}}}^3 \sin(k_{n_x} x) \sin(k_{n_y} y) \sin(k_{n_z} z),
\]
where
\[
	k_{n_x} = \frac{(n_x + 1) \pi}{a}, \qquad
	k_{n_y} = \frac{(n_y + 1) \pi}{a}, \qquad
	k_{n_z} = \frac{(n_z + 1) \pi}{a},
\]
and \(n_x, n_y, x_z = 0, 1, 2, \dots\). Then we have
\[
	E_n = \frac{\hbar^2}{2m} \vec k^2 = \frac{\hbar^2 \pi}{2m} [
		(n_x + 1)^2 +
		(n_y + 1)^2 +
		(n_z + 1)^2
	].
\]
However, note that this system is degenerate (in the sense that a particular energy does not correspond to a unique wavefunction). This can be seen easily by rearranging the dimensions.

\subsubsection{Harmonic Oscillator}
Let \(V(\vec r) = \half m\omega^2 (x^2 + y^2 + z^2) = \half m \omega^2 \vec r^2\). Let's again guess the solution \(\phi(x, y, z) = \phi_x(x) \phi_y(y) \phi_z(z)\). Recalling that in one dimension we have
\[
	E_n = \hbar \omega \parens*{n + \half},
\]
we have the solution in three-dimensions
\[
	E_{n_x, n_y, n_z} = \hbar\omega \parens*{n_x + n_y + n_z + \frac{3}{2}}.
\]
This again yields degeneracies. This is called the \vocab{isotropic harmonic oscillator}. To remedy some of this, we will modify the potential to be
\[
	V(\vec r) = \half m (\omega_x^2 x^2 + \omega_y^2 y^2 + \omega_z^2 z^2).
\]

\subsubsection{Radial Potentials}
Sometimes our potential will be a function of the distance from the origin, i.e.\ we can write \(V(\vec r) = V(\abs{\vec r}) = V(r)\). In this case it makes sense to use spherical coordinates. (Recall that we use \((r, \varphi)\) for the \(xy\)-plane and \(\theta\) for angle between \(\vec r\) and the \(z\)-axis). We have the Laplacian
\[
	\nabla^2 = \frac{1}{r^2} \pdv{}{r} \parens*{r^2 \pdv{}{r}} + \frac{1}{r^2 \sin \theta} \pdv{}{\theta} \parens*{\sin \theta \pdv{}{\theta}} + \frac{1}{r^2 \sin^2 \theta} \pddv{}{\varphi},
\]
which we will use in the Schr\"odinger equation.


\end{document}
