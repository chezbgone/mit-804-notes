\documentclass{scrartcl}
\usepackage{chez}

\begin{document}
\section{March 18, 2019}
\subsection{One-dimensional Systems}
\subsubsection{Finite Square Well}
Consider the finite square well potential:
\[
	V(x) = \begin{cases}
		-V_0 & x \in [-a, a] \\
		0 & \text{otherwise},
	\end{cases}
\]
where \(V_0\) is some positive constant.
\begin{center}
	\begin{asy}
		import graph;
		import cse5;

		real xmin = -2, xmax = -xmin;
		real ymin = -2, ymax = 0.25;
		real eps = 0.05;

		real x0 = 1, V0 = 1.5;
		real V(real x){ return abs(x) < x0 ? -V0 : 0; }

		D((xmin, 0)--(xmax, 0), gray(0.7), Arrows(TeXHead));
		D((0, ymin)--(0, ymax), gray(0.7), Arrows(TeXHead));
		D(graph(V, xmin, xmax));

		real E = -V0/2;
		D((xmin, E)--(xmax, E), n_blue+linetype("4 4"));
		MP("E < V_0", (xmax, E), dir(90), n_blue);

		MP("a", (x0, 0), dir(90));
		MP("-a", (-x0, 0), dir(90));
		MP("V_0", (x0, -V0), dir(0));
	\end{asy}
\end{center}

To solve this system, suppose we have a constant potential \(V(x) = V_0\) in some region. Then the time independent Schr\"odinger equation is
\[
	\psi''(x) = \frac{2m}{\hbar^2} (V_0 - E) \psi(x).
\]

Let's gain some intuition about the solution first.

If \(E > V_0\), then this is classically allowed, and we have the solutions \(\sin kx\) and \(\cos kx\), where \(k = \frac{\sqrt{2m(E - V_0)}}{\hbar}\). If \(E - V_0\) is large, then \(k\) is large, and \(\lambda = 2\pi/k\) is small. Similarly, if \(E - V_0\) is small, then \(\lambda\) is large.

If \(E < V_0\), then this is classically forbidden, and we have the solutions \(e^{\alpha x}\) and \(e^{-\alpha x}\), where \(\alpha = \frac{\sqrt{2m(E - V_0)}}{\hbar}\). Then, like before, if \(E - V_0\) is large, then we have \(\alpha\) is large, yielding a quick exponential function. Similarly, if \(E - V_0\) is small, then we have a slowly changing exponential function.

To actually solve this, let's consider a simpler case \(V\) that we can use to form a solution to the finite square well problem. Consider the potential
\[
	V(x) = \begin{cases}
		V_L & x < a \\
		V_r & x > a,
	\end{cases}
\]
where \(a \in \RR\).
\begin{center}
	\begin{asy}
		import graph;
		import cse5;

		real xmin = 0, xmax = 2.5;
		real ymin = -0.7, ymax = 1.6;
		real eps = 0.05;

		real a = 1, VL = 1.3, VR = -0.45;
		real V(real x){ return x < a ? VL : VR; }

		D((xmin, 0)--(xmax, 0), gray(0.7), Arrows(TeXHead));
		D((0, ymin)--(0, ymax), gray(0.7), Arrows(TeXHead));
		D((eps, VL)--MP("V_L", (-eps, VL), dir(180)), gray(0.7));
		D((eps, VR)--MP("V_R", (-eps, VR), dir(180)), gray(0.7));
		D((a, eps)--MP("a", (a, -eps), dir(-45)), gray(0.7));

		D(graph(V, xmin, xmax));

	\end{asy}
\end{center}

Consider the solutions \(\psi(x)\) of this potential. We know that \(\psi(x)\) is continuous. We will show that \(V(x)\) is continuous if \(V(x)\) is finite.
\begin{align*}
	\lim_{\Delta \to 0} \brackets{\psi'(a + \Delta) - \psi'(a - \Delta)}
		&= \lim_{\Delta \to 0} \int_{a - \Delta}^{a + \Delta} \dv{}{x} \psi'(x) \,dx \\
\intertext{By the Schr\"odinger equation,}
		&= \lim_{\Delta \to 0} \int_{a - \Delta}^{a + \Delta} \frac{2m}{\hbar^2} (V(x) - E) \psi(x) \,dx.
\end{align*}
If \(V(x)\) is finite, then we have that \(\frac{2m}{\hbar^2} [V(x) - E] \psi(x)\). Then, we can write
\[
	\abs*{\int_{a - \Delta}^{a + \Delta} \frac{2m}{\hbar^2} [V(x) - E] \psi(x) \,dx} < c \cdot 2\Delta
\]
for some constant \(c\) and for all \(\Delta\), so we have
\[
	\lim_{\Delta \to 0} \brackets{\psi'(a + \Delta) - \psi'(a - \Delta)} = 0,
\]
and \(\psi'\) is continuous for finite \(V(x)\).

Now suppose that \(V(x)\) is infinite. If we have a hard wall \(V_L = \infty\), then we have that
\[
	\lim_{\Delta \to 0} \int_{a - \Delta}^{a + \Delta} \frac{2m}{\hbar^2} (V(x) - E) \psi(x) \,dx \neq 0,
\]
and therefore \(\psi'\) is discontinuous. If we have a delta function \(\psi(x) = V_0 \delta(x - a)\), then
\begin{align*}
	\lim_{\Delta \to 0} \int_{a - \Delta}^{a + \Delta} \frac{2m}{\hbar^2} (V(x) - E) \psi(x) \,dx
		&= \lim_{\Delta \to 0} \brackets*{
			\int_{a - \Delta}^{a + \Delta} \frac{2m}{\hbar^2} V_0 \delta(x - a) \psi(x) \,dx
			- \int_{a - \Delta}^{a + \Delta} \frac{2m}{\hbar^2} E \psi(x) \,dx
		} \\
\intertext{Note that the second term goes to \(0\), so this equals}
		&= \lim_{\Delta \to 0} \frac{2m}{\hbar^2} V_0 \int_{a - \Delta}^{a + \Delta} \delta(x - a) \delta(x) \,dx = \frac{2m}{\hbar^2} V_0 \psi(a).
\end{align*}
Therefore, \(\psi'\) is discontinuous, and moreover there is a difference of \(\frac{2m}{\hbar^2} V_0 \psi(a)\) at the point \(a\).

\iffalse
\subsubsection{Harmonic Oscillator}
Consider the harmonic oscillator, where the potential is
\[
	V(x) = \half kx^2.
\]
\begin{center}
	\begin{asy}
		import graph;
		import cse5;

		real xmin = -2, xmax = -xmin;
		real ymin = -0.25, ymax = 2;

		real k = 0.7;
		real V(real x){ return 0.5 * k * x^2; }

		D((xmin, 0)--(xmax, 0), gray(0.7), Arrows(TeXHead));
		D((0, ymin)--(0, ymax), gray(0.7), Arrows(TeXHead));
		D(graph(V, xmin, xmax));
	\end{asy}
\end{center}
This is interesting because this is the first order Taylor expansion of any small oscillation.
\fi

\iffalse
\subsubsection{Shooting Method Practice}
Let's try to draw a wavefunction for the following potential and the given energy level.
\begin{center}
	\begin{asy}
		import cse5;
		import graph;

		real xmin = 0, xmax = 10.5;
		real ymin = 0, ymax = 5;
		real x1 = 2, x2 = 7;

		D((xmin, 0)--(xmax, 0), gray(0.6), EndArrow(TeXHead));
		D((0, ymin)--(0, ymax), gray(0.6), EndArrow(TeXHead));

		real V(real x){ return (0.612x^2 - 2.112x + 2.618)/(x - 1/2); }
		D(graph(V, 0.8, 10.2));

	\end{asy}
\end{center}
\fi




\end{document}
