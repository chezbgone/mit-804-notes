\documentclass{scrartcl}
\usepackage{chez}

\begin{document}
\section{May 08, 2019}
\subsection{Angular Part of Schr\"odinger Equation}
Recall that when we had the time independent Schr\"odinger equation, we let \(\psi(r, \theta, \varphi) = R(r) \cdot Y(\theta, \varphi)\), and this yields
\begin{align*}
	\frac{1}{R} \dv{}{r} \parens*{r^2 \dv{R}{r}} - \frac{2mr^2}{\hbar^2} [V(r) - E] &= \ell(\ell + 1) \\
	\frac{1}{Y} \brackets*{\frac{1}{\sin \theta} \pdv{}{\theta} \parens*{\sin\theta \pdv{Y}{\theta}} + \frac{1}{\sin^2\theta} \pddv{Y}{\varphi}} &= -\ell(\ell + 1).
\end{align*}
We solved this for \(\ell = 0\), i.e.\ for spherically symmetric solutions. Now suppose that \(\ell\) is any nonnegative integer. Let
\[
	Y(\theta, \varphi) = \Theta(\theta) \Phi(\varphi).
\]
We then get
\begin{align*}
	\frac{1}{\Theta} \brackets*{\sin\theta \dv{}{\theta} \parens*{\sin \theta \dv{\Theta}{\theta}} + \ell(\ell + 1) \sin^2 \theta} + \frac{1}{\Phi} \ddv{\Phi}{\varphi} = 0.
\end{align*}
The first term only depends on \(\theta\), and the second term only depends on \(\varphi\), so both must be a constant. Let the constant be \(m^2\). We get
\begin{gather*}
	\frac{1}{\Theta} \brackets*{\sin\theta \dv{}{\theta} \parens*{\sin \theta \dv{\Theta}{\theta}} + \ell(\ell + 1) \sin^2 \theta} = m^2 \\
	\frac{1}{\Phi} \ddv{\Phi}{\varphi} = -m^2.
\end{gather*}

The second equation gives us \(\Phi(\varphi) = \exp(i m \varphi)\), implying that \(m = 0, 1, 2, \dots\) since \(\Phi\) has period \(2\pi\).

The first equation is hard to solve. We know (from math) that the solution is
\[
	\Theta(\theta) = A P_{\ell m}(\cos \theta),
\]
where \(P_{\ell m}\) are the \vocab{associated Legendre function}, defined by
\[
	P_{\ell m}(x) = (1 - x^2)^{\abs{m}/2} \dnv{}{x}{\abs m} P_\ell(x),
\]
where \(P_\ell\) is the \(\ell\)th \vocab{Legendre polynomial}, defined by
\[
	P_\ell(x) = \frac{1}{2^\ell \ell!} \dnv{}{x}{\ell} (x^2 - 1)^\ell.
\]

Overall, this gives the \vocab{spherical harmonics} \(Y_{\ell m}(\theta, \varphi) = A \exp(i m \varphi) P_{\ell m}(\cos \theta)\). Here are some values of \(P_{\ell m}\) for small \(\ell\) and \(m\).
\begin{align*}
	P_{1, 1}(\cos\theta) &= \sin\theta &
		P_{1, 0}(\cos\theta) &= \cos\theta
\end{align*}
\begin{align*}
	P_{2, 2}(\cos\theta) &= 3\sin^2\theta &
		P_{2, 1}(\cos\theta) &= 3\sin\theta \cos\theta &
		P_{2, 0}(\cos\theta) &= \half (3 \cos^2\theta - 1).
\end{align*}

\subsubsection{Relationship between \texorpdfstring{\(Y_{\ell m}\)}{Ylm} and \texorpdfstring{\(\hat L^2\)}{L2}, \texorpdfstring{\(\hat L_z\)}{Lz}}
Writing \(\hat L^2\) in spherical coordinates, we have
\[
	\hat L^2 = -\hbar^2 \brackets*{\frac{1}{\sin\theta} \pdv{}{\theta} \parens*{\sin\theta \pdv{}{\theta}} + \frac{1}{\sin^2 \theta} \pddv{}{\varphi}}.
\]
We know that \(\psi_{\ell m}\) is an eigenvalue for \(\hat L^2\), so we have
\begin{align*}
	\hat L^2 \psi_{\ell m} &= \hbar^2 \ell (\ell + 1) \psi_{\ell m} \\
	-\hbar^2 \brackets*{
		\frac{1}{\sin\theta} \pdv{}{\theta} \parens*{\sin\theta \pdv{\psi_{\ell m}}{\theta}}
		+ \frac{1}{\sin^2 \theta} \pddv{\psi_{\ell m}}{\varphi}
	} &= \hbar^2 \ell(\ell + 1) \psi_{\ell m} \\
	\frac{1}{Y_{\ell m}} \brackets*{
		\frac{1}{\sin\theta} \pdv{}{\theta} \parens*{\sin\theta \pdv{\psi_{\ell m}}{\theta}}
		+ \frac{1}{\sin^2 \theta} \pddv{\psi_{\ell m}}{\varphi}
	} &= -\ell(\ell + 1).
\end{align*}
This is precisely the equation for the angular momentum, so the eigenvalues are \(\psi_{\ell m} = Y_{\ell m}\).

\subsection{Radial Part of Schr\"odinger Equation}
Recall that we let \(u(r) = r \cdot R(r)\). Then we have
\[
	-\frac{\hbar^2}{2m} \ddv{u}{r} + \brackets*{V + \frac{\hbar^2}{2m} \frac{\ell(\ell + 1)}{r^2}} u = Eu.
\]
We call \(V_{\text{eff}} = V + \frac{\hbar^2}{2m} \frac{\ell(\ell + 1)}{r^2}\) the \vocab{effective potential}, where the second term of the effective potential is called the \vocab{centrifugal term}. The centrifugal term is nonzero for \(\ell > 0\).

We can rewrite our equation as
\[
	-\frac{\hbar^2}{2m} \ddv{u}{r} + V_{\text{eff}} u = Eu,
\]
but it is difficult to solve without knowing a specific \(V_{\text{eff}}\).




\end{document}

